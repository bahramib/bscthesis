% ! TeX root = thesis_en.tex

\chapter{Conclusion}
\label{ch:sum}

This thesis demonstrated the problem of unchecked return values, to which problem it gave a solution in form of a statistical method without
actually modifying the source code of said unchecked return values. This solution was made as an implementation in the Clang-Tidy pattern
matching based static analysis library, however the statistical method would fail because of the way Clang-Tidy checkers work, separately
for each translation unit. This leads to invalid statistics. In order to fix this, I have enhanced the infrastructure and created the
option of multiphase analysis, that will collect the needed information per translation unit, accumulate these and make a project-level set
of information, and finally emit diagnostics with this newly compacted information. The finished checker was accomplished using this new
infrastructure. This checker method can obviously run without the use of multiphase analysis, because the new infrastructure is made to be
backwards compatible with the single phase mode, but the new infrastructure was vital for the statistics of the checker. This proved the
checker to be an excellent example for the usage and utility of the multiphase mode.

\section{Future Work}

Previously in \cref{sec:eval}, I talked about ignoring a set of functions whose total amount of calls do not exceed a given value. This could
potentially be implemented into the checker as an option, so that the user could give the checker the number, and the checker will not give
warnings to a function call if the function's amount of calls is less than said number. It would make the results clearer in some cases.

The first possible future work ignores functions based on their statistics, but ignoring based on function property could be useful too.
The checker should categorically ignore some functions. There are also functions whose return value can not always be checked. A classic
example would be the \mintinline{CPP}{<<} operator or \mintinline{CPP}{std::append}, since the first will always return the \mintinline{CPP}{stream}, and the latter
will always return \mintinline{CPP}{*this} for continuous use of the same function. \mintinline{CPP}{std::cout << "1" << "2" << "3" << std::endl;} will
always be unchecked in the end, since the last \lstinline{<<} can not be checked with merit and without creating another unchecked return value.

The hash generation for the YAML filename does not respect multiple compilations of the same file with different compile flags. This is something
that could be fixed in the future as well.

\section*{Acknowledgements}

I would like to express my deepest appreciation to my supervisor, Richárd Szalay for his invaluable patience and feedback, and without whom
I could not have completed this thesis in time. I am also grateful to my professor, Dr. Zoltán Porkoláb, who generously provided his knowledge,
expertise and inspiration. Furthermore I would highlight Aaron Ballman's (Intel Corporation) and Gábor Márton's (Ericsson AB.) insights during
the open-source code review process.
